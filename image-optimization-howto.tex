\documentclass{beamer}

\usepackage[utf8]{inputenc}
\usepackage[T1]{fontenc}
\usepackage{setspace}
\usepackage{color}
\usepackage{listings}
\usepackage{hyperref}
\usepackage{graphicx}

\usepackage[scaled]{beramono}
\newcommand\Small{\fontsize{9}{9.2}\selectfont}
\newcommand\LSTfont{\Small\ttfamily}

\usetheme{Dresden}

\title{Image Optimization in Rails}
\author{Constantin Hofstetter}
\subtitle{For an up-to-date version and discussions see \href{https://github.com/consti/image-optimization-howto}{https://github.com/consti/image-optimization-howto}}
\date{\today}

\begin{document}

\lstset{ %
language=Ruby,                % choose the language of the code
basicstyle=\LSTfont,       % the size of the fonts that are used for the code
stepnumber=1,                   % the step between two line-numbers. If it is 1 each line will be numbered
numbersep=5pt,                  % how far the line-numbers are from the code
backgroundcolor=\color{white},  % choose the background color. You must add \usepackage{color}
showspaces=false,               % show spaces adding particular underscores
showstringspaces=false,         % underline spaces within strings
showtabs=false,                 % show tabs within strings adding particular underscores
frame=false,           % adds a frame around the code
tabsize=2,          % sets default tabsize to 2 spaces
captionpos=b,           % sets the caption-position to bottom
breaklines=true,        % sets automatic line breaking
breakatwhitespace=false,    % sets if automatic breaks should only happen at whitespace
escapeinside={\%*}{*)},          % if you want to add a comment within your code
keywordstyle=\color{blue},
stringstyle=\color{red},
commentstyle=\color{green},
morecomment=[l][\color{magenta}]{\#}
}

\begin{frame}
\titlepage
\end{frame}

\begin{frame}{Reduce file size: Static Assets}
\begin{block}{ImageOptim}
Reduce image file sizes significantly by using multiple optimization (lossless compression) libraries packaged into one tool.
Available as:
\begin{itemize}
  \item \href{https://imageoptim.com}{GUI Application}
  \item \href{https://github.com/JamieMason/ImageOptim-CLI}{CLI Application}
  \item \href{https://github.com/plasticine/middleman-imageoptim}{Extension for Middleman}
\end{itemize}
\end{block}
\begin{block}{ImageAlpha}
Convert PNG24 images with transparent background to PNG8+alpha images to reduce file size.
\href{http://pngmini.com/}{Available here}.
\end{block}
\end{frame}

\begin{frame}[fragile]{Reduce file size: Static Assets}
\begin{block}{ImageOptim CLI as pre-commit hook}
Automatically optimize images that are committed to your git repo:
\begin{lstlisting}
# in your_project/.git/hooks/pre-commit
images=$(git diff --exit-code --cached --name-only --diff-filter=ACM -- '*.png' '*.jpg')
$(exit $?) || echo $images | imageoptim && git add $images
\end{lstlisting}
\end{block}
\end{frame}


\begin{frame}[fragile]{Reduce file size: Paperclip uploads}
\begin{block}{Using \href{http://imagemagick.org/}{\lstinline{imagemagick}} parameters}
Add options to the \lstinline{convert_options} parameter:
\begin{itemize}
  \item \lstinline{strip}: remove EXIF data
  \item \lstinline{quality PERCENTAGE}: compress image data (not lossless)
  \item \lstinline{interlace METHOD}: make the image progressive
\end{itemize}
Example:
\begin{lstlisting}
# In your model
has_attached_file :avatar, {
  styles: { thumb: '60x60#' },
  convert_options: {
    all: '-strip -quality 70 -interlace Line'
  }
}
\end{lstlisting}
\end{block}
\end{frame}

\begin{frame}[fragile]{Reduce file size: Paperclip uploads}
\begin{block}{Using \lstinline{paperclip-optimizer} processor}
The \href{https://github.com/janfoeh/paperclip-optimizer}{\lstinline{paperclip-optimizer}} gem uses the \href{https://github.com/toy/image_optim}{\lstinline{image_optim}} library
instead of \lstinline{imagemagick} to yield better optimization results.
\end{block}
\end{frame}

\begin{frame}[fragile]{Reduce file size: Paperclip uploads}

Using \lstinline{paperclip-optimizer} processor

\begin{lstlisting}
# Add paperclip-optimizer gem
# Install image_optim binaries / image_optim_bin gem (HEROKU)
has_attached_file :avatar, {
  styles: {
    thumb: {
      geometry: '60x60#',
      paperclip_optimizer: {
        # add compression (not lossless) to optimization (lossless)
        jpegoptim: { max_quality: 70 },
        # Disable unavailable binaries
        svgo: false
      }
    }
  },
  # We still need to strip EXIF and make the image progressive
  convert_options: { all: '-strip -interlace Line' },
  processors: [:paperclip_optimizer]
}
\end{lstlisting}
\end{frame}

\begin{frame}[fragile]{Reduce file size: Paperclip uploads}
\begin{block}{Multiple geometries}
Make sure to include the \lstinline{:thumbnail} processor if you want to save your image in different geometries.
\begin{lstlisting}
# in your model
has_attached_file :avatar, {
  ...
  processors: [:thumbnail, :paperclip_optimizer]
}
\end{lstlisting}
\end{block}
\end{frame}


\begin{frame}{Image delivery: Best practices}
\begin{block}{Store images on file storage web service}
Use a file storage web service such as Amazon Simple Storage Service (S3) to store image uploads and static assets.
\end{block}
\begin{block}{Use a CDN}
Content Delivery Network: images and static assets are served from the closest edge server, based on the clients location, e.g. Amazon CloudFront
\end{block}
\begin{block}{Allow Clients to cache images and static assets}
Use \lstinline{Cache-Control} or \lstinline{Expires} headers.
\end{block}
\end{frame}

\begin{frame}[fragile]{File storage web service: Paperclip uploads}
\begin{lstlisting}
# in your Gemfile
gem 'aws-sdk'

# in your Paperclip config:
PAPERCLIP_STORAGE_OPTIONS = {
  :storage => :s3,
  :s3_host_name => 'REMOVE_THIS_LINE_IF_UNNECESSARY',
  :s3_credentials => {
    :bucket => 'S3_BUCKET_NAME'
  }
}

# in your aws.yml
development:
  access_key_id: AWS_ACCESS_KEY_ID
  secret_access_key: AWS_SECRET_KEY_ID

production:
  access_key_id: AWS_ACCESS_KEY_ID
  secret_access_key: AWS_SECRET_KEY_ID
\end{lstlisting}
\end{frame}

\begin{frame}[fragile]{File storage web service: Static Assets}

Add \lstinline{asset_sync} to your Gemfile.

\begin{lstlisting}
# In your asset_sync.rb
s3_config = YAML.load_file("#{Rails.root}/config/s3.yml")[Rails.env]

config.aws_access_key_id = s3_config['access_key_id']
config.aws_secret_access_key = s3_config['secret_access_key']
config.fog_directory = s3_config['bucket']
# config.fog_region = 'eu-west-1'
config.gzip_compression = true
\end{lstlisting}

Then run:

\begin{lstlisting}
# RAILS_GROUPS=assets will make asset sync run in verbose mode
bundle exec rake assets:precompile RAILS_GROUPS=assets
\end{lstlisting}
\end{frame}


\begin{frame}[fragile]{Expiration: Best practices}
\begin{block}{Expires or Cache-Control?}
Expires is a fixed future date, Cache-Control is the amount of time (in seconds) the client is allowed to cache the object.
\end{block}
\begin{block}{Expires or Cache-Control for CloudFront \& S3}
CloudFront accepts both values \lstinline{Expires} and \lstinline{Cache-Control}. If you specify values both for \lstinline{Cache-Control} and for \lstinline{Expires}, CloudFront uses only the value of \lstinline{Cache-Control}.

\lstinline{Cache-Control} is the recommended way to set expiration.
\end{block}
\end{frame}


\begin{frame}[fragile]{Expiration: Paperclip uploads}
\begin{block}{Cache-Control in Paperclip}
Include \lstinline{Cache-Control} in your \lstinline{s3_headers}-Paperclip config:
\end{block}
\begin{lstlisting}
# in your Paperclip config:
PAPERCLIP_STORAGE_OPTIONS = {
  ...,
  s3_headers: lambda { |_attachment|
    {
      'Cache-Control' => "public, max-age=#{ 1.month.to_i }"
    }
  }
}
\end{lstlisting}
\end{frame}


\begin{frame}[fragile]{Expiration: Static Assets}
\begin{lstlisting}
# in your asset_sync.rb
config.custom_headers = {
  # for fonts and svg, you have to set Access-Control-Allow-Origin
  '.*\.[svg|woff|ttf|eot]' => {
    'Access-Control-Allow-Origin' => '*',
    cache_control: "public, max-age=#{ 1.month.to_i }",
    expires: nil
  },
  # Everything else, set cache-control to max-age 1 month
  '.*' => {
    cache_control: "public, max-age=#{ 1.month.to_i }",
    expires: nil
  }
}
\end{lstlisting}
\end{frame}

\begin{frame}{Other Optimizations: Domain Sharding}
Clients have a max-limit of connections to a single host (e.g. modern browsers up to ten connections).

Domain Sharding helps by splitting images and static assets onto different hosts (i.e. domains that point to the same resource).
\begin{block}{SPDY and HTTP 2.0}
SPDY and HTTP 2.0 (which integrates SPDY) allow clients to keep connections open to stream files.

Domain Sharding will actually hurt the performance of HTTP 2.0 and SPDY compatible browsers (unnecessary domain lookups).
\end{block}
\end{frame}

\begin{frame}[fragile]{Other Optimizations: Domain Sharding}
Create different CloudFront hosts that point to the same S3 bucket (where you store images/static assets).
\begin{lstlisting}
# in your app config (e.g. production.rb):
config.assets.cdn_hosts = %w(xxxxxx01.cloudfront.net
                             xxxxxx02.cloudfront.net
                             xxxxxx03.cloudfront.net)
\end{lstlisting}
\end{frame}

\begin{frame}[fragile]{Domain Sharding: Paperclip uploads}
We calculate the asset host through the \lstinline{update_at} in seconds of the file (to make sure the file always gets served from the same host).
\begin{lstlisting}
# in Paperclip config:
PAPERCLIP_STORAGE_OPTIONS = {
  ...,
  s3_host_alias: lambda { |attachment|
    Rails.configuration.assets.cdn_hosts[
      attachment.updated_at.to_i % Rails.configuration.assets.cdn_hosts.count
    ]
  },
  url: ':s3_alias_url'
}
\end{lstlisting}
\end{frame}


\begin{frame}[fragile]{Domain Sharding: Static Assets}
We calculate the asset host through the MD5 sum of the file (to make sure the file always gets served from the same host).
\begin{lstlisting}
# in your app config (e.g. production.rb):
config.action_controller.asset_host = Proc.new { |source|
  'https://' + Rails.configuration.assets.cdn_hosts[
    Digest::MD5.hexdigest(source).to_i(16) % Rails.configuration.assets.cdn_hosts.count
  ]
}
\end{lstlisting}
\end{frame}

\end{document}
